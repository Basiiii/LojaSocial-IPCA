\documentclass[a4paper, 12pt]{article} % Font size (can be 10pt, 11pt or 12pt) and paper size (remove a4paper for US letter paper)
\usepackage[portuguese]{babel}

\usepackage[protrusion=true,expansion=true]{microtype} % Better typography
\usepackage{graphicx} % Required for including pictures
\usepackage{wrapfig} % Allows in-line images
\usepackage{pdflscape} %Allows landscape oriented pages
\usepackage{xcolor}

\usepackage{hyperref}%Required for hyperlink references
\hypersetup{
	colorlinks=true, % Ativa links coloridos em vez de caixas ao redor
	linkcolor=black, % Cor dos links internos
	citecolor=black, % Cor das citações
	urlcolor=black   % Cor dos links externos (Jira, por exemplo)
}

\usepackage{mathpazo} % Use the Palatino font
\usepackage[T1]{fontenc} % Required for accented characters
\linespread{1.05} % Change line spacing here, Palatino benefits from a slight increase by default
\usepackage{float}
\usepackage[backend=bibtex,style=numeric]{biblatex}
\addbibresource{references.bib} % Nome do arquivo de referências
\makeatletter
\renewcommand\@biblabel[1]{\textbf{#1.}} % Change the square brackets for each bibliography item from '[1]' to '1.'
\renewcommand{\@listI}{\itemsep=0pt} % Reduce the space between items in the itemize and enumerate environments and the bibliography

\renewcommand{\maketitle}{
\begin{titlepage}
\begin{center}
\vspace*{1cm}
% \includegraphics[width=0.35\textwidth]{../images/logo-no-bg.png}\\[1cm] % Logo
{\Huge\textbf{Loja Social IPCA}}\\[0.5cm] % Main Title
{\Large Projeto 50+10}\\[2cm] % Subtitle
{\large \textsc{
		Enrique Rodrigues Nº28602 \\
		José Alves Nº27967 \\
		Diogo Machado Nº26042 \\
		Carlos Barreiro Nº20360
	}}\\[0.5cm] % Authors
{\textit{Instituto Politécnico do Cávado e do Ave}}\\[1.5cm] % Institution

{\large\textsc{\textbf{Regulamento Interno}}}\\[1.5cm]

% {\large 06 de outubro de 2025} % Date
{\large \today} % Date
\vfill
% \textbf{Keywords:} lorem, ipsum, dolor, sit amet, lectus % Keywords
\end{center}
\end{titlepage}
}
\makeatother

%------------------------------------------------------------------------------------

\begin{document}
\maketitle % Print the title section

\newpage
\renewcommand{\contentsname}{Índice}
\tableofcontents

\newpage

\section{Regulamento Interno do Grupo}

O grupo vai reger-se pelas seguintes regras, que têm como objetivo garantir a organização, a cooperação e o cumprimento das responsabilidades por parte de todos os elementos:

\begin{itemize}
	\item Serão atribuídas tarefas a cada um dos elementos;
	\item As datas e horários das reuniões podem ser alteradas, desde que respeitem os prazos estipulados, de forma a evitar faltas. Qualquer alteração só poderá ser feita mediante aprovação do restante grupo;
	\item É permitido faltar a reuniões, desde que prévia e devidamente justificado e as tarefas a apresentar estejam concluídas.
\end{itemize}
\subsection{Sistema de Avaliação Interno}

Neste sistema de avaliação interno, todos os membros da equipa começam com um total de 20 valores, e as penalizações diferem de ligeiras a mais graves, sendo as ligeiras correspondentes a uma perda de 0,5 valor e as graves a uma perda de 1 valor:

\begin{itemize}
	\item Falta de presença nas reuniões semanais, sem aviso prévio, e a não realização da tarefa atribuída resultam numa perda de 1 valor;
	\item A não realização da tarefa atribuída dentro do prazo estipulado implica uma penalização de 1 valor;
	\item Ausência sem comunicação prévia, mas com a tarefa atribuída concluída, resulta numa perda de 0,5 valor;
	\item Atrasos recorrentes na entrega de tarefas ou nas reuniões podem implicar uma perda de 0,5 valor por cada ocorrência;
	\item A falta de participação ativa nas discussões de grupo poderá levar a uma penalização de 0,5 valor;
	\item O desrespeito pelas opiniões dos colegas ou comportamentos que comprometam a harmonia da equipa implicam uma perda de 1 valor.
\end{itemize}

\newpage
\section{Disposições Finais}

\begin{itemize}
	\item Este regulamento pode ser revisto ou alterado mediante consenso de todos os membros do grupo;
	\item Os casos omissos neste regulamento serão resolvidos pelo grupo, em conjunto, de forma democrática;
	\item O presente documento entra em vigor após a sua aprovação por todos os elementos do grupo.
\end{itemize}

\end{document}