\documentclass[a4paper, 12pt]{article} % Font size (can be 10pt, 11pt or 12pt) and paper size (remove a4paper for US letter paper)
\usepackage[portuguese]{babel}

\usepackage[protrusion=true,expansion=true]{microtype} % Better typography
\usepackage{graphicx} % Required for including pictures
\usepackage{wrapfig} % Allows in-line images
\usepackage{pdflscape} %Allows landscape oriented pages
\usepackage{xcolor}

\usepackage{hyperref}%Required for hyperlink references
\hypersetup{
	colorlinks=true, % Ativa links coloridos em vez de caixas ao redor
	linkcolor=black, % Cor dos links internos
	citecolor=black, % Cor das citações
	urlcolor=black   % Cor dos links externos (Jira, por exemplo)
}

\usepackage{mathpazo} % Use the Palatino font
\usepackage[T1]{fontenc} % Required for accented characters
\linespread{1.05} % Change line spacing here, Palatino benefits from a slight increase by default
\usepackage{float}
\usepackage[backend=bibtex,style=numeric]{biblatex}
\addbibresource{references.bib} % Nome do arquivo de referências
\makeatletter
\renewcommand\@biblabel[1]{\textbf{#1.}} % Change the square brackets for each bibliography item from '[1]' to '1.'
\renewcommand{\@listI}{\itemsep=0pt} % Reduce the space between items in the itemize and enumerate environments and the bibliography

\renewcommand{\maketitle}{
\begin{titlepage}
\begin{center}
\vspace*{1cm}
% \includegraphics[width=0.35\textwidth]{../images/logo-no-bg.png}\\[1cm] % Logo
{\Huge\textbf{Loja Social IPCA}}\\[0.5cm] % Main Title
{\Large Projeto 50+10}\\[2cm] % Subtitle
{\large \textsc{
		Enrique Rodrigues Nº28602 \\
		José Alves Nº27967 \\
		Diogo Machado Nº26042 \\
		Carlos Barreiro Nº20360
	}}\\[0.5cm] % Authors
{\textit{Instituto Politécnico do Cávado e do Ave}}\\[1.5cm] % Institution

{\large\textsc{\textbf{Atas de Reunião}}}\\[1.5cm]

{\large 06 de outubro de 2025} % Date
%{\large \today} % Date
\vfill
% \textbf{Keywords:} lorem, ipsum, dolor, sit amet, lectus % Keywords
\end{center}
\end{titlepage}
}
\makeatother

%------------------------------------------------------------------------------------

\begin{document}
\maketitle % Print the title section

\newpage
\renewcommand{\contentsname}{Índice}
\tableofcontents

\newpage
\section{Reunião 22 Setembro 2025}

\subsection*{Contexto}
\textbf{Local}: IPCA – Escola Superior de Tecnologia - Laboratório de Redes
\newline\textbf{Hora}: 11h00
\newline\textbf{Data}: 22 de Setembro de 2025
\newline\textbf{Participantes}: Carlos Barreiro, Diogo Machado, Enrique Rodrigues, José Alves, Doutora Mónica (SASIPCA)

\subsection*{Ordem de Trabalhos}
\begin{itemize}
	\item 1- Apresentação do Projeto
	\item 2 - Funcionamento Atual do Processo
\end{itemize}

\subsection*{Desenvolvimento}

\subsubsection*{1 - Apresentação do Projeto}

	Proposta de desenvolvimento de uma aplicação móvel (\textit{Android}) que possa ser utilizada pelos funcionários dos \textbf{SASIPCA }para fazer a gestão de \textit{stocks} dos produtos existentes na sua Loja Social, bem como gestão das entregas.
	Deve também ser feita uma página web, a incluir no site do IPCA, onde seja possível ver a proporção de \textit{stocks} existente em armazém.
	
	\begin{itemize}
		\item \textbf{Aplicação móvel (uso interno dos SAS)}:
		
		\begin{itemize}
			\item Gestão de beneficiários (base de dados de estudantes apoiados).
			\item Gestão de inventário (entradas, saídas, validade dos bens).
			\item Calendarização de apoios (agendamentos e lembretes).
			\item Seleção e registo de bens entregues (com atualização automática de stock).
			\item Alertas de validade e relatórios de produtos.
		\end{itemize}
		
		\item \textbf{Website informativo (comunidade académica)}:
		
		\begin{itemize}
			\item Gráfico com stock em tempo real (por categorias e quantidades disponíveis).
			\item Secção para doações/contribuições pessoais.
			\item Área de notícias sobre campanhas e divulgação de bens recolhidos.
		\end{itemize}
	\end{itemize}
	
\subsubsection*{2 - Funcionamento Atual do Processo}	
	\begin{itemize}
		\item Os bens são recolhidos em \textbf{campanhas} ou \textbf{doações pontuais};
		\item Os bens recolhidos (quantidades e categorias) são registados numa folha excel;
		\item Aquando da preparação do saco para entrega, apenas se regista na folha excel a quantidade entregue, que é abatida ao \textit{stock};
		\item O que cada beneficiário leva é registado num \textbf{caderno pessoal};
		\item A entrega é agendada por \textit{email};
		\item Os \textbf{produtos perecíveis} são guardados em prateleiras, de acordo com o \textbf{trimestre} em que a validade acaba (prateleira marcada com \textit{post-its});
		\item \textbf{Dois dias} antes da entrega, é enviada uma notificação por \textit{email} ao beneficiário;
		\item É feita uma\textbf{ pré-seleção dos produtos} antes da entrega (fim da tarde anterior ou manhã do dia da entrega).
	\end{itemize}
	


\subsection*{Deliberação e próximos passos}
\begin{itemize}
	\item O grupo compromete-se a estruturar os requisitos funcionais com base nas observações acima.
	\item Na próxima reunião, serão discutidas ideias que vão de encontro às necessidades mencionadas.
\end{itemize}

\subsection*{Encerramento}
\textbf{Encerramento da reunião:} 12h30
\newline \textbf{Elaborada por:} José Alves
\newline \textbf{Aprovada por:} Enrique Rodrigues

\newpage

\section{Reunião 26 Setembro 2025}

\subsection*{Contexto}
\textbf{Local}: IPCA – Escola Superior de Tecnologia - Laboratório de Redes
\newline\textbf{Hora}: 11h00
\newline\textbf{Data}: 26 de Setembro de 2025
\newline\textbf{Participantes}: Diogo Machado, Enrique Rodrigues, José Alves

\subsection*{Ordem de Trabalhos}
\begin{itemize}
	\item 1 - Preparação das questões a fazer à Dra. Mónica
	\item 2 - Discussão dos Requisitos Funcionais e Não Funcionais baseados nessas questões
	
\end{itemize}

\subsection*{Desenvolvimento}

\subsubsection*{1 - Preparação das questões a fazer à Dra. Mónica }

No próximo dia 3 de Outubro de 2025, em nova reunião com a Dra. Mónica, fazer as seguintes questões:

\begin{itemize}
	\item Tem de se registar quem doa os produtos?
	\item Há registo de quem recebe?
	\item Já existe uma BD de estudantes apoiados pelos SAS e de doadores?
	\begin{itemize}
		\item Se sim, podemos ter acesso à sua estrutura?
	\end{itemize}
	\item São necessários graus de permissão? Ou todos os funcionários podem fazer tudo?
	\item Que tipos de grupos/categorias de bens existem? 
	\item Quem falha um levantamento, pode reagendar?
		\begin{itemize}
			\item É possível alterar a data prevista da recolha?
			\item O funcionário recebe notificação em caso de alteração da data?
		\end{itemize}
	\item Quem agenda o levantamento de bens? O colaborador ou o beneficiário? Se for o beneficiário, onde o faz?
	\item Existem diferentes escalões de beneficiários?
		\begin{itemize}
			\item Se sim, o que os distingue? Quantidade de bens? Número de recolhas mensais?
		\end{itemize}
	\item Em relação ao \textbf{site}:
	\begin{itemize}
		\item É possível fazer uma doação anónima?
		\item A aba a implementar no site dos SAS é na barra do topo ou na lateral? (Aparentemente, na lateral)
	\end{itemize}
	
\end{itemize}

\subsubsection*{Discussão dos Requisitos Funcionais e Não Funcionais baseados nessas questões}

\begin{itemize}
	\item \textbf{Requisitos Funcionais (RF)}
		\begin{itemize}
			\item \textbf{Gestão de Stock}
				\begin{itemize}
					\item RF1.1 - Adicionar Item: O sistema deve permitir ao administrador registar novos itens, incluindo nome, categoria, quantidade e validade opcional.
					\item RF1.2 - Editar Item: O sistema deve permitir ao administrador atualizar quantidades e outros detalhes dos itens existentes.
					\item RF1.3 - Ver Stock Atual: O sistema deve permitir ao administrador consultar uma lista completa de todos os itens disponíveis, com filtros por categoria.
					\item RF1.4 - Gestão de Categorias: O sistema deve permitir criar, editar e eliminar categorias para organizar os itens.
					\item RF1.5 - Scan de Entrada: O sistema deve permitir registar a entrada de um item através do scan do código de barras ou QR, associando-o automaticamente ao stock.
					\item RF1.6 - Validade Manual: Após o scan do item em entrada, o sistema deve permitir registar manualmente a data de validade. 
				\end{itemize}
			\item \textbf{Gestão de Doações}
				\begin{itemize}
					\item RF2.1 - Registar Doação: O sistema deve permitir ao administrador registar doações, indicando doador, itens doados, quantidade e data.
					\item RF2.2 - Associar ao Stock: O sistema deve atualizar automaticamente o stock disponível quando um item é registado como doação.
				\end{itemize}
			\item Gestão de Recolhas
				\begin{itemize}
					\item RF3.1 - Criar Marcação de Recolha: O sistema deve permitir criar um agendamento de recolha indicando data, hora e itens.
					\item RF3.2 - Editar/Cancelar Marcação: O sistema deve permitir alterar ou cancelar uma recolha agendada.
					\item RF3.3 - Confirmar Recolha: O sistema deve permitir marcar a recolha como concluída.
					\item RF3.4 - Scan de Saída: O sistema deve permitir registar a saída de um item através do scan do código de barras ou QR, removendo-o do stock.
					\item RF3.5 - Data Manual de Saída: Após o scan do item em saída, o sistema deve permitir registar manualmente a data da recolha.
					\item RF3.6 - Visualização de Recolhas: O sistema deve permitir consultar as recolhas futuras ou passadas numa lista ou calendário, filtrando por data, tipo de item ou estado.
				\end{itemize}
		\end{itemize}
	\item \textbf{Requisitos Não Funcionais (RNF)}
		\begin{itemize}
			\item RNF1 - Segurança: Apenas utilizadores autenticados com perfil de administrador podem adicionar, editar ou remover itens e gerir doações/recolhas.
			\item RNF2 - Performance: O sistema deve processar scans e atualizações de stock de forma eficiente.
			\item RNF3 - Usabilidade: A interface de scan de código de barras/QR deve ser intuitiva e rápida, com mínima necessidade de dados manuais.
			\item RNF4 - Compatibilidade: O sistema deve funcionar em dispositivos Android.
			\item RNF5 - Fiabilidade: Todos os registos de entrada e saída devem ser persistentes e auditáveis.
		\end{itemize}
\end{itemize}

\subsection*{Deliberação e próximos passos}
\begin{itemize}
	\item Iniciar elaboração do relatório de projeto e finalização do Regulamento Interno do grupo.
	\item Discussão de novas ideias de funcionalidades e possibilidades de arquiteturas do sistema.
\end{itemize}

\subsection*{Encerramento}
\textbf{Encerramento da reunião:} 12h40
\newline \textbf{Elaborada por:} José Alves
\newline \textbf{Aprovada por:} Enrique Rodrigues

\newpage

\section{Reunião 29 Setembro 2025}

\subsection*{Contexto}
\textbf{Local}: IPCA – Escola Superior de Tecnologia - Laboratório de Redes
\newline\textbf{Hora}: 11h00
\newline\textbf{Data}: 29 de Setembro de 2025
\newline\textbf{Participantes}: Carlos Barreiro, Diogo Machado, Enrique Rodrigues

\subsection*{Ordem de Trabalhos}
\begin{itemize}
	\item 1 - Revisão das questões a fazer à Dra. Mónica
	\item 2 - Início da Elaboração do Relatório
	
\end{itemize}

\subsection*{Desenvolvimento}

\subsubsection*{1 - Revisão das questões a fazer à Dra. Mónica}

As questões preparadas na reunião anterior foram validadas.

Surge a ideia de utilizar a leitura de Códigos de Barras para facilitar a gestão de stocks.

\subsubsection*{2 - Início da Elaboração do Relatório}
Iniciou-se o relatório de projeto e o Estudo de Viabilidade.

\subsection*{Deliberação e próximos passos}
\begin{itemize}
	\item Concluir o Estudo de Viabilidade.
	\item Preparação da reunião com Dra. Mónica.
\end{itemize}

\subsection*{Encerramento}
\textbf{Encerramento da reunião:} 12h25
\newline \textbf{Elaborada por:} Diogo Machado
\newline \textbf{Aprovada por:} Enrique Rodrigues

\newpage

\section{Reunião 03 Outubro 2025}

\subsection*{Contexto}
\textbf{Local}: IPCA – Escola Superior de Tecnologia - Laboratório de Redes
\newline\textbf{Hora}: 11h00
\newline\textbf{Data}: 03 de Outubro de 2025
\newline\textbf{Participantes}: Diogo Machado, Enrique Rodrigues, José Alves, Dra. Mónica (SASIPCA)

\subsection*{Ordem de Trabalhos}
\begin{itemize}
	\item 1 - Validação de Requisitos e Perguntas à Dra. Mónica.
	
\end{itemize}

\subsection*{Desenvolvimento}

\subsubsection*{1 - Validação de Requisitos e Perguntas à Dra. Mónica.}

\begin{itemize}
	\item Tem de se registar quem doa os produtos? \textcolor{blue}{Não. Apenas se é originário de Recolha de Campanha Interna/Externa ou Doação Esporádica}
	\item Há registo de quem recebe? \textcolor{blue}{Sim!}
	\item Já existe uma BD de estudantes apoiados pelos SAS e de doadores? \textcolor{blue}{Não existe BD nenhuma.}
	\item São necessários graus de permissão? Ou todos os funcionários podem fazer tudo? \textcolor{blue}{Todos os Funcionários dos SAS podem fazer tudo.}
	\item Que tipos de grupos/categorias de bens existem? \textcolor{blue}{arroz, massa, higiene pessoal, higiene da casa, etc.}
	\item Quem falha um levantamento, pode reagendar? \textcolor{blue}{Sim, por e-mail.}
	\begin{itemize}
		\item É possível alterar a data prevista da recolha? \textcolor{blue}{Sim, por e-mail.}
		\item O funcionário recebe notificação em caso de alteração da data? \textcolor{blue}{É o funcionário quem altera o agendamento na app.}
	\end{itemize}
	\item Quem agenda o levantamento de bens? O colaborador ou o beneficiário? Se for o beneficiário, onde o faz? \textcolor{blue}{O agendamento é feito por e-mail ou sms entre o colaborador dos SAS e o beneficiário. É o colaborador quem agenda a recolha na app.}
	\item Existem diferentes escalões de beneficiários? \textcolor{blue}{Não. A única distinção que existe é \textbf{Estudante} ou \textbf{Não Estudante}}
	\item Em relação ao \textbf{site}:
	\begin{itemize}
		\item É possível fazer uma doação anónima? \textcolor{blue}{Sim. Os bens são entregues nos locais de recolha das campanhas ou diretamente nos SAS, pelo que é só deixar lá os bens, sem ser necessário recolha de dados pessoais.} 
		\item A aba a implementar no site dos SAS é na barra do topo ou na lateral? (Aparentemente, na lateral) \textcolor{blue}{Sim, aba lateral.}
	\end{itemize}
	
\end{itemize}

\subsection*{Deliberação e próximos passos}
\begin{itemize}
	\item Continuação da elaboração do relatório.
	\item Início da elaboração de diagramas.
	\item Conclusão do Estudo de Viabilidade.
\end{itemize}

\subsection*{Encerramento}
\textbf{Encerramento da reunião:} 12h45
\newline \textbf{Elaborada por:} José Alves
\newline \textbf{Aprovada por:} Enrique Rodrigues

\newpage

\section{Reunião 06 Outubro 2025}

\subsection*{Contexto}
\textbf{Local}: IPCA – Escola Superior de Tecnologia - Laboratório de Redes
\newline\textbf{Hora}: 11h00
\newline\textbf{Data}: 06 de Outubro de 2025
\newline\textbf{Participantes}: Carlos Barreiro, Diogo Machado, Enrique Rodrigues, José Alves

\subsection*{Ordem de Trabalhos}
\begin{itemize}
	\item 1 - Avançar no relatório
	\item 2 - Elaboração de Diagramas de Contexto e Arquitetura de Sistema
	
\end{itemize}

\subsection*{Desenvolvimento}

\subsubsection*{1 - Avançar no relatório}

Concluiu-se o Estudo de Viabilidade.
Introduziram-se os novos dados no relatório.

\subsubsection*{2 - Elaboração de Diagramas de Contexto e Arquitetura de Sistema}
Foram feitos os diagramas de contexto e arquitetura do sistema.
\newline Definiu-se a utilização de:
\begin{itemize}
	\item Landing Page da LojaSocial: \textbf{NextJS}
	\item App LojaSocial: \textbf{Kotlin Android}
	\item Necessidades de Backend (Storage, Auth e SMTP/Email): \textbf{Firebase}
\end{itemize}

\subsection*{Deliberação e próximos passos}
\begin{itemize}
	\item Continuação da elaboração do relatório.
	\item Definição do Processo de Negócio e elaboração do respetivo diagrama BPMN.
\end{itemize}

\subsection*{Encerramento}
\textbf{Encerramento da reunião:} 12h40
\newline \textbf{Elaborada por:} José Alves
\newline \textbf{Aprovada por:} Enrique Rodrigues

\newpage

\section{Reunião 10 Outubro 2025}

\subsection*{Contexto}
\textbf{Local}: IPCA – Escola Superior de Tecnologia - Laboratório de Redes
\newline\textbf{Hora}: 11h00
\newline\textbf{Data}: 10 de Outubro de 2025
\newline\textbf{Participantes}: Carlos Barreiro, Diogo Machado, Enrique Rodrigues, José Alves

\subsection*{Ordem de Trabalhos}
\begin{itemize}
	\item 1 - Avançar no relatório
	
\end{itemize}

\subsection*{Desenvolvimento}

\subsubsection*{1 - Avançar no relatório}

Introduziram-se os dados da arquitetura de sistema no relatório.

\subsection*{Deliberação e próximos passos}
\begin{itemize}
	\item Terminar BPMN.
	\item Preparação da Primeira Entrega do projeto.
\end{itemize}

\subsection*{Encerramento}
\textbf{Encerramento da reunião:} 15h35
\newline \textbf{Elaborada por:} José Alves
\newline \textbf{Aprovada por:} Enrique Rodrigues

\newpage

\section{Reunião 13 Outubro 2025}

\subsection*{Contexto}
\textbf{Local}: IPCA – Escola Superior de Tecnologia - Laboratório de Redes
\newline\textbf{Hora}: 11h00
\newline\textbf{Data}: 13 de Outubro de 2025
\newline\textbf{Participantes}: Diogo Machado, Enrique Rodrigues, José Alves

\subsection*{Ordem de Trabalhos}
\begin{itemize}
	\item 1 - Discussão e Desenvolvimento do diagrama BPMN
	
\end{itemize}

\subsection*{Desenvolvimento}

\subsubsection*{1 - Discussão e Desenvolvimento do diagrama BPMN}

Confirmaram-se requisitos e processos de negócio a incluir no diagrama BPMN a desenvolver.

\subsection*{Deliberação e próximos passos}
\begin{itemize}
	\item Terminar BPMN.
	\item Finalizar Primeira Entrega do projeto.
\end{itemize}

\subsection*{Encerramento}
\textbf{Encerramento da reunião:} 12h30
\newline \textbf{Elaborada por:} José Alves
\newline \textbf{Aprovada por:} Enrique Rodrigues

\newpage
\section{Reunião 17 Outubro 2025}

\subsection*{Contexto}
\textbf{Local}: IPCA – Escola Superior de Tecnologia - Laboratório de Redes
\newline\textbf{Hora}: 14h00
\newline\textbf{Data}: 17 de Outubro de 2025
\newline\textbf{Participantes}: Carlos Barreiro, Diogo Machado, Enrique Rodrigues, José Alves, Prof. Patrícia Leite

\subsection*{Ordem de Trabalhos}
\begin{itemize}
	\item 1 - Alteração de Requisitos (Reunião com cliente)
	\item 2 - Conclusão Diagrama BPMN
	\item 3 - Conclusão da Calendarização
	
\end{itemize}

\subsection*{Desenvolvimento}

\subsubsection*{1 - Alteração de Requisitos (Reunião com cliente)}

Após novas informações dadas pelo cliente, temos agora de implementar as seguintes funcionalidades:

\begin{itemize}
	\item login/registo/recuperação de acessos
	\item perfil de beneficiário
	\item candidatura
	\item monitorização da candidatura (tipo estado da encomenda)
	\item pedidos (tipo lista de supermercado)
	\item disponibilidade de levantamento
	\item canal de suporte da aplicação
\end{itemize}

Dado que estas informações chegaram muito em cima da hora, os requisitos funcionais/não funcionais que daqui advenham, só serão descritos numa próxima entrega, que não a prevista para dia 20/10/2025.

\subsubsection*{2 - Conclusão Diagrama BPMN}

Concluiu-se a criação do diagrama BPMN a incluir na entrega de dia 20/10/2025.

\subsubsection*{ 3 - Conclusão da Calendarização}
Concluiu-se a criação do calendário a incluir na entrega de dia 20/10/2025.

\subsection*{Deliberação e próximos passos}
\begin{itemize}
	\item Proceder à Primeira Entrega do projeto.
	\item Reavaliar os Requisitos Funcionais e Não Funcionais, para que contemplém as novas funcionalidades pedidas.
\end{itemize}

\subsection*{Encerramento}
\textbf{Encerramento da reunião:} 15h37
\newline \textbf{Elaborada por:} José Alves
\newline \textbf{Aprovada por:} Enrique Rodrigues

\newpage

\section{Reunião 20 Outubro 2025}

\subsection*{Contexto}
\textbf{Local}: IPCA – Escola Superior de Tecnologia - Laboratório de Redes
\newline\textbf{Hora}: 11h00
\newline\textbf{Data}: 20 de Outubro de 2025
\newline\textbf{Participantes}: Carlos Barreiro, Diogo Machado, Enrique Rodrigues, José Alves

\subsection*{Ordem de Trabalhos}
\begin{itemize}
	\item 1 - Finalizar dos documentos a entregar na primeira fase
	\item 2 - Validar os documentos a entregar. 
	
\end{itemize}

\subsection*{Desenvolvimento}

\subsubsection*{1 - Finalizar dos documentos a entregar na primeira fase}

Terminaram-se os relatórios, diagramas e restantes documentos a entregar.

\subsubsection*{2 - Validar os documentos a entregar}

Validaram-se todos dos documentos a entregar e procedeu-se à entrega dos mesmos.\newline


\textbf{NOTA IMPORTANTE:} Houve uma alteração nos requisitos que, dado o prazo desta primeira entrega, não serão descritos na documentação da mesma, mas devem ser corrijidos para a próxima fase. São eles:

\begin{itemize}
	\item login/registo/recuperação de acessos
	\item perfil de beneficiário
	\item candidatura
	\item monitorização da candidatura (tipo estado da encomenda)
	\item pedidos (tipo lista de supermercado)
	\item disponibilidade de levantamento
	\item canal de suporte da aplicação
\end{itemize}

\subsection*{Deliberação e próximos passos}
\begin{itemize}
	\item Preparar apresentação do projeto aos SAS.
	\item Ajuste da documentação entregue para contemplar os novos requisitos.
\end{itemize}

\subsection*{Encerramento}
\textbf{Encerramento da reunião:} 12h45
\newline \textbf{Elaborada por:} José Alves
\newline \textbf{Aprovada por:} Enrique Rodrigues

\newpage

\section{Reunião 27 Outubro 2025}

\subsection*{Contexto}
\textbf{Local}: IPCA – Escola Superior de Tecnologia - Laboratório de Redes
\newline\textbf{Hora}: 11h00
\newline\textbf{Data}: 27 de Outubro de 2025
\newline\textbf{Participantes}: Carlos Barreiro, Diogo Machado, Enrique Rodrigues, José Alves,  \textbf{Dra. Juliana (FASA)}

\subsection*{Ordem de Trabalhos}
\begin{itemize}
	\item 1 - \textit{Workshop} sobre Gestão de Conflitos 
	
\end{itemize}

\subsection*{Desenvolvimento}

\subsubsection*{1 - \textit{Workshop} sobre Gestão de Conflitos com Dra. Juliana}

Assistimos a um \textit{workshop} sobre gestão de conflitos., adquirindo competências para lidar melhor com adversidades, melhorar o poder de negociação e chegar a consensos.

\subsection*{Deliberação e próximos passos}
\begin{itemize}
	\item Continuar preparação da apresentação do projeto aos SAS.
	\item Terminar a alteração da documentação entregue para contemplar os novos requisitos.
\end{itemize}

\subsection*{Encerramento}
\textbf{Encerramento da reunião:} 12h50
\newline \textbf{Elaborada por:} José Alves
\newline \textbf{Aprovada por:} Enrique Rodrigues

\newpage

\section{Reunião 31 Outubro 2025}

\subsection*{Contexto}
\textbf{Local}: IPCA – Escola Superior de Tecnologia - Laboratório de Redes
\newline\textbf{Hora}: 14h00
\newline\textbf{Data}: 31 de Outubro de 2025
\newline\textbf{Participantes}: Carlos Barreiro, Diogo Machado, Enrique Rodrigues, José Alves, \textbf{Dra. Mónica (SASIPCA)}, \textbf{Dra. Carla (SASIPCA)}

\subsection*{Ordem de Trabalhos}
\begin{itemize}
	\item 1 - Apresentação da Primeira proposta de Sistema 
	
\end{itemize}

\subsection*{Desenvolvimento}

\subsubsection*{1 - Apresentação da Primeira proposta de Sistema}

Foi apresentada a Primeira Versão do sistema a implementar, com as seguintes notas do cliente:

\begin{itemize}
	\item Não há grandes restrições quanto ao que fazer no \textit{site}. Podemos implementar à nossa maneira.
	\item \textbf{Canal de Suporte} -> Formação e ajuda sobre como funciona a \textit{app} para \textit{comunidade académica} e \textit{staff}.
	\item É necessário um perfil de \textit{admin}.
	\item Caso seja \textbf{estudante bolseiro}, já não é necessário entregar mais documentação. Só é necessário preencher o formulário de candidatura.
		\begin{itemize}
			\item \textit{É beneficiário de algum tipo de Apoio Social? Se Sim, qual? De que valor?}
		\end{itemize}
	\item Quando um estudante deixa de ser estudante (conclusão de curso/cancelamento de matrícula/etc) perde \textbf{automaticamente} o apoio. Para ficar bem feito e prático:
		\begin{itemize}
			\item Acesso através do \textit{login} ca Comunidade IPCA para facilitar registo e acesso à \textit{app}.
			\item Isto permite também ao sistema validar (semanalmente, por exemplo) a lista de beneficiários e perceber se alguém já não é estudante/funcionário.
		\end{itemize}
	\item Entregas só com \textbf{agendamento de dia} (\textbf{SEM HORA MARCADA}), para não ser limitador.
	\item É possível fazer uma doação 100\% anónima (não é guardado registo, nem para os técnicos).
	\item Registo à parte das entregas em caso de emergência (entregas pontuais).
	\item Chegando a Agosto, os beneficiários não perdem necessariamente o apoio.
	\item \textbf{Estados do Pedido de Apoio}:
		\begin{itemize}
			\item Submetido
			\item Pendente de Análise
			\item Em Análise
			\item Aguarda Documentos Adicionais
			\item Aprovado/Rejeitado
		\end{itemize}
	\item Caso o \textit{Stock} não seja suficiente para satisfazer os pedidos, deve abrir-se um canal de comunicação com o Estudante.
		\begin{itemize}
			\item Quais as suas necessidades principais no momento?
		\end{itemize}
\end{itemize}

\subsection*{Deliberação e próximos passos}
\begin{itemize}
	\item Terminar a alteração da documentação entregue para contemplar os novos requisitos.
\end{itemize}

\subsection*{Encerramento}
\textbf{Encerramento da reunião:} 15h50
\newline \textbf{Elaborada por:} José Alves
\newline \textbf{Aprovada por:} Enrique Rodrigues

\newpage

\section{Reunião 3 Novembro 2025}

\subsection*{Contexto}
\textbf{Local}: IPCA – Escola Superior de Tecnologia - Laboratório de Redes
\newline\textbf{Hora}: 14h00
\newline\textbf{Data}: 31 de Novembro de 2025
\newline\textbf{Participantes}: Carlos Barreiro, Diogo Machado, Enrique Rodrigues, José Alves

\subsection*{Ordem de Trabalhos}
\begin{itemize}
	\item 1 - Revisão do progresso geral do projeto
	\item 2 - Atribuição das novas tarefas 
	
\end{itemize}

\subsection*{Desenvolvimento}

\subsubsection*{1 - Revisão do progresso geral do projeto}

Avaliação 

\subsection*{Deliberação e próximos passos}
\begin{itemize}
	\item Terminar a alteração da documentação entregue para contemplar os novos requisitos.
\end{itemize}

\subsection*{Encerramento}
\textbf{Encerramento da reunião:} 15h50
\newline \textbf{Elaborada por:} José Alves
\newline \textbf{Aprovada por:} Enrique Rodrigues

\newpage

\end{document}