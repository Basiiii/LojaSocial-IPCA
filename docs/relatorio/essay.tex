\documentclass[a4paper, 12pt]{article} % Font size (can be 10pt, 11pt or 12pt) and paper size (remove a4paper for US letter paper)
\usepackage[portuguese]{babel}

\usepackage[protrusion=true,expansion=true]{microtype} % Better typography
\usepackage{graphicx} % Required for including pictures
\usepackage{wrapfig} % Allows in-line images
\usepackage{pdflscape} %Allows landscape oriented pages

\usepackage{hyperref}%Required for hyperlink references
\hypersetup{
	colorlinks=true, % Ativa links coloridos em vez de caixas ao redor
	linkcolor=black, % Cor dos links internos
	citecolor=black, % Cor das citações
	urlcolor=black   % Cor dos links externos (Jira, por exemplo)
}

\usepackage{mathpazo} % Use the Palatino font
\usepackage[T1]{fontenc} % Required for accented characters
\linespread{1.05} % Change line spacing here, Palatino benefits from a slight increase by default
\usepackage{float}
\usepackage[backend=bibtex,style=numeric]{biblatex}
\addbibresource{references.bib} % Nome do arquivo de referências
\makeatletter
\renewcommand\@biblabel[1]{\textbf{#1.}} % Change the square brackets for each bibliography item from '[1]' to '1.'
\renewcommand{\@listI}{\itemsep=0pt} % Reduce the space between items in the itemize and enumerate environments and the bibliography

\renewcommand{\maketitle}{
\begin{titlepage}
\begin{center}
\vspace*{1cm}
% \includegraphics[width=0.35\textwidth]{../images/logo-no-bg.png}\\[1cm] % Logo
{\Huge\textbf{Loja Social IPCA}}\\[0.5cm] % Main Title
{\Large Projeto 50+10}\\[2cm] % Subtitle
{\large \textsc{
		Enrique Rodrigues Nº28602 \\
		José Alves Nº27967 \\
		Diogo Machado Nº26042 \\
		Carlos Barreiro Nº20360
	}}\\[0.5cm] % Authors
{\textit{Instituto Politécnico do Cávado e do Ave}}\\[1.5cm] % Institution
%{\large 9 de março de 2025} % Date
{\large \today} % Date
\vfill
% \textbf{Keywords:} lorem, ipsum, dolor, sit amet, lectus % Keywords
\end{center}
\end{titlepage}
}
\makeatother

%------------------------------------------------------------------------------------

\begin{document}
\maketitle % Print the title section

\newpage
\renewcommand{\contentsname}{Índice}
\tableofcontents

\newpage
\renewcommand{\listfigurename}{Lista de Figuras}
\listoffigures

\newpage
\section{Introdução}

Em tempos de desafios socioeconómicos crescentes, iniciativas de solidariedade como a \textbf{Loja Social} do \textbf{Instituto Politécnico do Cávado e do Ave (IPCA)} assumem um papel fundamental no apoio à comunidade académica. O projeto não se limita à entrega de bens essenciais: pretende também modernizar a gestão da Loja Social e fortalecer a participação da comunidade nas iniciativas solidárias do IPCA.

O documento apresenta o projeto \textbf{50+10}, que tem como objetivo criar duas soluções digitais complementares: uma aplicação móvel, destinada ao uso interno pelos \textbf{Serviços de Ação Social (SAS)}, e um website informativo, acessível a toda a \textbf{comunidade académica}. A aplicação permitirá gerir de forma prática os beneficiários, o inventário, a calendarização de entregas e os alertas de validade, enquanto o website proporcionará uma visão clara do stock disponível e incentivará a participação da comunidade em doações e campanhas solidárias.

Além da descrição do sistema e das funcionalidades propostas, o documento detalha os \textbf{objetivos de negócio}, os \textbf{requisitos funcionais e não funcionais} do sistema, e ainda os \textbf{regulamentos internos do grupo}, definindo responsabilidades, organização e critérios de avaliação. Assim, o documento oferece uma visão geral do projeto, permitindo entender não apenas como o sistema funcionará, mas também como contribuirá para a comunidade.

\newpage

%------------------------------------------------------------------------------------
\section{Contextualização do Projeto}

O Instituto Politécnico do Cávado e do Ave (IPCA) criou a \textbf{Loja Social} para apoiar a comunidade académica em situações de necessidade, disponibilizando bens alimentares, de higiene e de limpeza, obtidos via doações ou campanhas internas.

Para otimizar a gestão e a resposta às necessidades, serão desenvolvidas duas soluções digitais:

\subsubsection*{Aplicação móvel (uso interno pelos SAS)}
Funcionalidades principais:
\begin{itemize}
	\item Gestão de beneficiários e inventário;
	\item Calendarização de apoios e lembretes;
	\item Atualização automática de stock;
	\item Alertas de validade.
\end{itemize}

\subsubsection*{Website informativo (acesso público)}
Funcionalidades principais:
\begin{itemize}
	\item Informações sobre doações e stock;
	\item Divulgação de campanhas e notícias.
\end{itemize}

Estas ferramentas aumentam a eficiência interna e promovem transparência e envolvimento da comunidade académica.

\begin{figure}[h!]
	\centering
	\includegraphics[width=1.0\textwidth]{../images/context-diagram.png}
	\caption{Diagrama de Contexto da Loja Social}
\end{figure}

\newpage
\section{Descrição do Negócio}
A Loja Social do IPCA surge como uma iniciativa de responsabilidade social do Instituto Politécnico do Cávado e do Ave, com a missão de apoiar a comunidade académica, em particular os estudantes em situação de maior vulnerabilidade económica e social. Este espaço funciona como um ponto de acolhimento e de distribuição gratuita de bens essenciais, tais como géneros alimentares, produtos de higiene pessoal e de limpeza, obtidos através de doações da comunidade, de empresas e de campanhas promovidas pelo próprio IPCA.  
A Loja Social constitui, assim, uma rede de partilha e solidariedade, contribuindo para a redução de desigualdades e para a promoção do bem-estar dos membros da instituição.

\section{Objetivos de Negócio}
O projeto em desenvolvimento tem como objetivo principal a criação de soluções digitais que tornem a gestão da Loja Social mais eficiente e transparente. Para tal, serão implementadas:
\begin{itemize}
	\item Uma \textbf{aplicação móvel}, de uso interno pelos colaboradores dos Serviços de Ação Social (SAS), que permitirá gerir beneficiários, inventário, calendários de distribuição, entregas e alertas de validade;
	\item Um \textbf{website informativo}, acessível a toda a comunidade académica, que disponibilizará informação sobre stocks em tempo real, campanhas de recolha de bens e instruções para efetuar doações.
\end{itemize}

Com estas ferramentas, pretende-se não apenas modernizar e agilizar a gestão operacional da Loja Social, mas também reforçar o envolvimento da comunidade académica nas práticas de solidariedade promovidas pelo IPCA.

\newpage
\newpage
\section{Descrição dos Interessados (Stakeholders)}

Os interessados, ou \textit{stakeholders}, correspondem a todas as entidades que, de forma direta ou indireta, influenciam ou são afetadas pelo desenvolvimento e utilização das soluções digitais da Loja Social do IPCA. A identificação destes grupos é essencial para assegurar que o sistema responde adequadamente às suas necessidades e expectativas.

\subsection{Serviços de Ação Social (SAS) do IPCA}
Os Serviços de Ação Social (SAS) do IPCA são responsáveis pela gestão da Loja Social e pela coordenação das atividades de apoio aos estudantes.  
Constituem os principais utilizadores da aplicação móvel, pretendendo dispor de uma ferramenta digital que permita uma gestão mais eficiente de beneficiários, inventário, campanhas e agendamentos de entregas.  
Estes serviços participam ativamente na definição de requisitos, validação de funcionalidades e acompanhamento da implementação do sistema.

\subsection{Beneficiários}
Os beneficiários são membros da comunidade académica que recebem apoio da Loja Social.
Embora não utilizem diretamente as aplicações, são impactados positivamente pela melhoria na organização e eficiência da gestão de stocks e do processo de distribuição de bens.
O sistema visa, desta forma, otimizar o controlo de inventário e assegurar uma gestão mais eficiente, transparente e equitativa dos recursos disponíveis.

\subsection{Doadores e Colaboradores Externos}
Incluem-se neste grupo as pessoas, empresas e organizações que contribuem com bens ou recursos para a Loja Social.  
Estes interessados procuram transparência na gestão das doações e acesso a informação atualizada sobre campanhas e necessidades da instituição.  
A sua interação ocorre através do \textbf{website informativo}, que disponibiliza dados sobre stocks, campanhas e instruções para efetuar doações.

\newpage
\section{Cronograma}

O cronograma apresentado descreve as atividades realizadas até ao momento no desenvolvimento do projeto, 
bem como as próximas ações planeadas, incluindo reuniões com o cliente e datas de entrega previstas. 
Este planeamento permite uma visão global do progresso do projeto, facilitando o acompanhamento das tarefas concluídas 
e a organização das etapas futuras, de forma a garantir o cumprimento dos prazos e a coordenação eficaz da equipa.

\begin{figure}[h!]
	\centering
	\includegraphics[width=1\textwidth]{../images/cronograma.png}
	\caption{Cronograma}
	\label{fig:cronograma}
\end{figure}

\newpage
\section{Requisitos Funcionais}
Esta secção apresenta os requisitos funcionais \textbf{(RFs)} que descrevem as funcionalidades a implementar no sistema para responder às necessidades dos utilizadores e da organização.

\begin{table}[H]
	\centering
	\renewcommand{\arraystretch}{1.3}
	\begin{tabular}{|c|p{12cm}|}
		\hline
		\textbf{RF 01} & O sistema deve permitir ao administrador registar e atualizar itens, incluindo nome, categoria, quantidade e validade opcional, garantindo a atualização automática do stock. \\
		\hline
		\textbf{RF 02} & O sistema deve permitir consultar a lista de itens disponíveis, com possibilidade de aplicar filtros por categoria e/ou validade. \\
		\hline
		\textbf{RF 03} & O sistema deve permitir a gestão de categorias (criação, edição e eliminação) para organizar os itens. \\
		\hline
		\textbf{RF 04} & O sistema deve permitir registar a entrada e saída de itens através do código de barras, associando data de validade e atualizando automaticamente o stock. \\
		\hline
		\textbf{RF 05} & O sistema deve permitir ao administrador registar e gerir campanhas, indicando nome e datas associadas. \\
		\hline
		\textbf{RF 06} & O sistema deve permitir agendar recolhas, especificando data e hora. \\
		\hline
		\textbf{RF 07} & O sistema deve permitir alterar, cancelar ou concluir recolhas previamente agendadas. \\
		\hline
		\textbf{RF 08} & O sistema deve disponibilizar uma lista ou calendário de recolhas futuras e passadas, com possibilidade de filtragem por data e/ou estado. \\
		\hline
		\textbf{RF 09} & O sistema deve permitir o acesso através de autenticação com PIN. \\
		\hline
		\textbf{RF 10} & O sistema deve permitir a extração de relatórios de informação sobre o stock. \\
		\hline
		\textbf{RF 11} & O sistema deve dar alertas de validade com um mês de antecedência. \\
		\hline
	\end{tabular}
	\caption{Requisitos Funcionais}
	\label{tab:requisitos_funcionais}
\end{table}

%------------------------------------------------------------------------------------

\newpage
\section{Requisitos Não Funcionais}

Os requisitos não funcionais \textbf{(RNFs)} definem como o sistema deve funcionar em termos de qualidade, desempenho, segurança e usabilidade, garantindo uma aplicação eficiente e confiável para os utilizadores.

\begin{table}[H]
	\centering
	\renewcommand{\arraystretch}{1.3}
	\begin{tabular}{|c|p{12cm}|}
		\hline
		\textbf{Código} & \textbf{Requisito Não Funcional} \\
		\hline
		\textbf{RNF01} & O sistema deve garantir autenticação segura do utilizador através de PIN, protegendo contra tentativas sucessivas de acesso indevido. \\
		\hline
		\textbf{RNF02} & O sistema deve processar scans e atualizações de stock de forma eficiente. \\
		\hline
		\textbf{RNF03} & A interface de scan de código de barras deve ser intuitiva e rápida, com mínima necessidade de dados manuais. \\
		\hline
		\textbf{RNF04} & O sistema deve funcionar em dispositivos Android. \\
		\hline
		\textbf{RNF05} & Todos os registos de entrada e saída devem ser persistentes e auditáveis. \\
		\hline
		\textbf{RNF06} & A interface deve ser responsiva e adaptável a diferentes tamanhos de ecrã e resoluções de dispositivos Android. \\
		\hline
		\textbf{RNF07} & O sistema deve apresentar mensagens de erro claras e informativas, sem expor informação técnica sensível. \\
		\hline
	\end{tabular}
	\caption{Requisitos Não Funcionais}
	\label{tab:requisitos_nao_funcionais}
\end{table}


%------------------------------------------------------------------------------------

\newpage

\section{Introdução aos Diagramas BPMN}

O BPMN (\textit{Business Process Model and Notation}) é uma notação padronizada para a modelação de processos de negócio. Permite representar graficamente as atividades, decisões e fluxos de um processo, facilitando a compreensão e comunicação entre diferentes intervenientes.  

Os diagramas BPMN ajudam a documentar, analisar e otimizar processos, proporcionando uma visão clara e estruturada das operações de uma organização.


\begin{figure}[h!]
	\centering
	\includegraphics[height=0.6\textheight]{../images/bpmn-diagram.png}
	\caption{Diagrama BPMN}
	\label{fig:diagrama-bpmn}
\end{figure}
\newpage
\section{Estudo de Viabilidade}

O estudo de viabilidade tem como objetivo analisar se o desenvolvimento das soluções propostas - uma aplicação móvel interna e uma página web informativa - é viável nas dimensões técnica, económica, operacional, legal e de mercado. Esta análise ajuda a identificar riscos, estimar custos e confirmar o alinhamento com os objetivos do IPCA e da Loja Social.

\subsection{Viabilidade Técnica}
A viabilidade técnica é considerada elevada, uma vez que o projeto pode ser implementado com tecnologias amplamente utilizadas e bem documentadas, como \textbf{Kotlin} para a aplicação móvel e \textbf{Next.js} para o website. A infraestrutura necessária pode ser baseada em \textbf{Firebase}, o que simplifica a escalabilidade e manutenção. Os requisitos técnicos levantados (gestão de stock, alertas, autenticação, etc.) são comuns em sistemas de gestão e não exigem recursos tecnológicos invulgares.

\subsection{Viabilidade Económica}
O custo estimado do projeto está limitado à alocação de tempo e recursos humanos dos membros do grupo, sem necessidade de investimento financeiro externo. Em contexto académico, não se prevê qualquer custo de licenciamento, dado que se poderá utilizar software de código aberto. A manutenção futura poderá ser assegurada por alunos de anos seguintes ou incorporada na gestão interna dos SAS, com custos reduzidos.

\subsection{Viabilidade Operacional}
O sistema será operado pelos colaboradores dos Serviços de Ação Social (SAS), que possuem conhecimento suficiente sobre os processos envolvidos. A aplicação será desenhada para ser intuitiva, com formação mínima. O website informativo será de fácil manutenção, podendo ser atualizado por técnicos do IPCA. Assim, a viabilidade operacional é também considerada alta.

\subsection{Viabilidade Legal}
O sistema estará em conformidade com a \textbf{Lei de Proteção de Dados (RGPD)}, garantindo que os dados pessoais dos beneficiários são tratados de forma segura e restrita. A aplicação incluirá autenticação e históricos de recolhas/entregas. Não se antevêem barreiras legais significativas à implementação das soluções.

\subsection{Viabilidade de Mercado / Social}
O projeto responde a uma necessidade real identificada pelos SAS e pela comunidade académica. Existe procura interna por uma gestão mais eficiente da Loja Social e por uma comunicação mais transparente com a comunidade. Sendo um projeto de cariz social e institucional, não depende de competitividade de mercado, o que reforça a sua viabilidade.

\subsection{Conclusão}
Com base na análise efetuada, conclui-se que o projeto é \textbf{totalmente viável} nas dimensões analisadas. As soluções digitais propostas contribuem para a modernização dos serviços sociais do IPCA e para o reforço do espírito de solidariedade na comunidade académica, com custos mínimos e forte impacto positivo.


\newpage
\section{Arquitetura do Sistema}

A figura \ref{fig:arquitetura} apresenta a arquitetura da aplicação, integrando a página web em Next.js, a aplicação Android em Kotlin e o backend em Firebase com Firestore, Storage, Auth e envio de emails via SMTP.

\begin{figure}[h!]
	\centering
	\includegraphics[width=1.1\textwidth]{../images/system-architecture.png}
	\caption{Arquitetura geral da aplicação.}
	\label{fig:arquitetura}
\end{figure}

A página web, desenvolvida em Next.js, serve como ponto de acesso público para a comunidade académica, permitindo visualizar o stock da Loja Social em tempo real, consultar informações sobre campanhas de doação e acompanhar notícias relacionadas. O Next.js fornece renderização server-side quando necessário, garantindo desempenho otimizado e carregamento rápido de conteúdo.

A aplicação móvel, construída em Kotlin para Android, é utilizada pelos Serviços de Ação Social (SAS) para gerir os beneficiários, o inventário de bens, a calendarização de entregas e alertas de validade.

%----------------------------------------------------------------------------------------
%	BIBLIOGRAPHY
%----------------------------------------------------------------------------------------
\nocite{*}

\printbibliography
%----------------------------------------------------------------------------------------

\end{document}